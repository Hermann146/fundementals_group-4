% Options for packages loaded elsewhere
\PassOptionsToPackage{unicode}{hyperref}
\PassOptionsToPackage{hyphens}{url}
\documentclass[
]{article}
\usepackage{xcolor}
\usepackage[margin=1in]{geometry}
\usepackage{amsmath,amssymb}
\setcounter{secnumdepth}{-\maxdimen} % remove section numbering
\usepackage{iftex}
\ifPDFTeX
  \usepackage[T1]{fontenc}
  \usepackage[utf8]{inputenc}
  \usepackage{textcomp} % provide euro and other symbols
\else % if luatex or xetex
  \usepackage{unicode-math} % this also loads fontspec
  \defaultfontfeatures{Scale=MatchLowercase}
  \defaultfontfeatures[\rmfamily]{Ligatures=TeX,Scale=1}
\fi
\usepackage{lmodern}
\ifPDFTeX\else
  % xetex/luatex font selection
\fi
% Use upquote if available, for straight quotes in verbatim environments
\IfFileExists{upquote.sty}{\usepackage{upquote}}{}
\IfFileExists{microtype.sty}{% use microtype if available
  \usepackage[]{microtype}
  \UseMicrotypeSet[protrusion]{basicmath} % disable protrusion for tt fonts
}{}
\makeatletter
\@ifundefined{KOMAClassName}{% if non-KOMA class
  \IfFileExists{parskip.sty}{%
    \usepackage{parskip}
  }{% else
    \setlength{\parindent}{0pt}
    \setlength{\parskip}{6pt plus 2pt minus 1pt}}
}{% if KOMA class
  \KOMAoptions{parskip=half}}
\makeatother
\usepackage{graphicx}
\makeatletter
\newsavebox\pandoc@box
\newcommand*\pandocbounded[1]{% scales image to fit in text height/width
  \sbox\pandoc@box{#1}%
  \Gscale@div\@tempa{\textheight}{\dimexpr\ht\pandoc@box+\dp\pandoc@box\relax}%
  \Gscale@div\@tempb{\linewidth}{\wd\pandoc@box}%
  \ifdim\@tempb\p@<\@tempa\p@\let\@tempa\@tempb\fi% select the smaller of both
  \ifdim\@tempa\p@<\p@\scalebox{\@tempa}{\usebox\pandoc@box}%
  \else\usebox{\pandoc@box}%
  \fi%
}
% Set default figure placement to htbp
\def\fps@figure{htbp}
\makeatother
\setlength{\emergencystretch}{3em} % prevent overfull lines
\providecommand{\tightlist}{%
  \setlength{\itemsep}{0pt}\setlength{\parskip}{0pt}}
\usepackage{bookmark}
\IfFileExists{xurl.sty}{\usepackage{xurl}}{} % add URL line breaks if available
\urlstyle{same}
\hypersetup{
  pdftitle={Speaker notes},
  pdfauthor={hads2, sba11, js1392, sad44, rp576},
  hidelinks,
  pdfcreator={LaTeX via pandoc}}

\title{Speaker notes}
\author{hads2, sba11, js1392, sad44, rp576}
\date{2025-12-10}

\begin{document}
\maketitle

\#\#Slide 1: Hello everyone. Today, we'll be presenting our analysis
titled `How Does Italy's Reduction of Carbon Emissions Compare to Its
Region, the USA, and China?' In this presentation, we'll explore key
trends, examine how Italy's progress aligns with major global countries,
and discuss what these differences mean for environmental policy and
future energy strategies. \#\#Slide 2: The objective is to assess
Italy's overall discharge in CO2 emissions and energy use. We compare
Italy with the EU, China and the US to understand its emission patterns.
We also examine sector wise per capita emissions and the role of
renewable energy over period. Finally, we review Italy's energy mix and
environmental policies to understand its progress towards
sustainability. \#\# Slide 3: This slide explains the overall process
for the project with justifications for the methods used. In the
beginning we investigated how Italy reduces their carbon emission
compared to other countries like EU, China and US using the metrics
renewable electricity share and emission. All date sources are from OWID
who provide reliable and trustworthy dataset which are accessible to the
public. We cleaned the data by filtering and renaming columns to
specific years which ensures fair comparisons. We created different
graphs like area chart, line chart and pie chart based on the question
\#\# Slide 4: The illustrated data highlights Italy's distinct
trajectory in global CO₂ emissions. Regionally, Italy has consistently
maintained a lower carbon footprint than the wider European Union
average, with both following a steady downward trend over the last three
decades.However, globally the contrast is dramatic. The United States
operates on a vastly larger scale, emitting nearly three times as much
CO₂ per person as Italy. Meanwhile, China follows a completely inverse
path. Fueled by rapid industrialization, China surged from low levels to
overtake Italy's declining figures, reaching nearly 9 tonnes per capita
today

\subsection{Slide 5:}\label{slide-5}

Italy's per-capita CO₂ emissions peaked in the mid-2000s, driven mainly
by electricity/heat and transport. Since then, all major
sectors---especially power generation---have declined, reflecting
cleaner energy, efficiency gains, and economic shifts. Transport and
buildings now make up a larger share of a smaller total. The long-term
downward trend highlights how policy, technology, and behavior shape
national emissions. This matters for the public because meeting EU
climate goals, stabilizing energy costs, and improving air quality all
depend on sustaining and accelerating these sector-level transitions.
\#\# slide 6: This chart presents line graphs of per capita carbon
dioxide emissions from various sectors, comparing Italy with several
major countries. Italy has mid level emissions across most sectors,
while the United States shows high emissions in all sectors except
industry. China's emissions rise sharply in the industry and electricity
sectors, and the European Union remains slightly above but close to
Italy. Overall, Italy shows a consistent decline after 2006 in most
sectors, indicating improvement. \#\# slide 7: This line graph showcases
the percentage of electricity produced from renewable energy sources. We
compare this with the EU, China and US from 1990 to 2022. Italy
generation of renewable electricity has increased with significant
acceleration beginning from 2008. Also, Italy outperforms the US and
China but is behind EU who show faster growth. The trends suggest all
countries are committed to transitioning to renewable electricity which
the graph shows meaning Italy progress towards more sustainable energy
is positive. The visual analyses Italy's issue of high emission to see
if they are transitioning towards cleaner energy. \#\# Slide 8: Italy
shows comparatively strong hydro and solar contributions, aligning
closely with EU renewable trends and reflecting public interest in
cleaner energy. It also uses far less coal and oil than the other
countries, giving it a cleaner fossil-fuel profile and supporting demand
for reduced environmental impact. However, despite producing more
renewable energy than the US and China---and a share similar to the
EU---Italy still relies heavily on gas for its overall energy
production. This dependence highlights ongoing environmental and
sustainability challenges the country must address. \#\# Slide 9: As you
can see regionally Italy occupies a unique position. It is the Circular
Economy leader, outperforming EU countries in resource efficiency, yet
it trails in renewable deployment speed due to bureaucratic hurdles.
While aligning with EU goals, Italy champions `tech neutrality,' keeping
biofuels viable alongside EVs. However, USA's has an incentive-based
approach to reduce emissions whereas Italy relies on regulatory `sticks'
like carbon pricing. Finally, compared to China, Italy faces a
democratic trade-off. Italy's emissions are falling unlike China's
surge, but its consensus-driven permitting is significantly slower than
China's rapid `command-and-control' infrastructure buildout. \#\# Slide
10: Italy shows a steady and sustained reduction in carbon emissions
compared with its peers. Its per capita emissions remain well below the
EU average and far below the USA and China. Strong growth in renewable
electricity and a cleaner energy mix,especially low coal use---support
this downward trend. Overall, Italy aligns closely with EU progress and
outperforms both the USA and China in terms of sustainability, though
further reductions in transport and buildings are essential for meeting
long-term climate goals. That is the end of our presentaion, thank you
for listening.

\end{document}
